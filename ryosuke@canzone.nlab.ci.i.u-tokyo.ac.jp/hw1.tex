\documentclass{article}
\begin{document}
\begin{center}
\section*{先端データ解析論 レポート第1回}
48196635 桑原亮介
\end{center}
\section{chapter1}
\subsection{宿題1}
\subsubsection{p(X=好,Y=眠)を求めよ.}
$P(X=$好$)=0.8$, $P(Y=$眠 $\mid$ $X=$好$)=0.25$より,\\
$P(Y=$眠 , $X=$好)=$P$(X=好)・$P(Y=$眠 $\mid$ $X=$好)$=0.2$(答)\\
\subsubsection{ p(Y=眠)を求めよ}
$P(Y=$眠)$=P(Y=$眠 $\mid$ $X=$好)$+P(Y=$眠 $\mid$ $X=$嫌) $=0.5$(答)\\
\subsubsection{ P(X=好|Y=眠) を求めよ.}
P(X=好|Y=眠)=$\frac{P(Y=\mbox{眠}\mid X=\mbox{好})P(X=\mbox{好})}{P(Y=\mbox{眠})}$ = $\frac{0.2}{0.5}$=0.4(答)
\subsubsection{ 確率と統計の好き嫌いと 授業中眠たい事は独立か?}
独立ではない。\\
P(Y=眠)・P(X=好) $\neq$ P(Y=眠 , X=好)(答)

\subsection{宿題2}
証明\\
\subsubsection{定数は期待値をとっても値は変わらない:\[E(c)=c\]}
E(c)=cE(1)=c (答)
\subsubsection{定数を足した期待値は,期待値に定数を足した
ものと等しい:\[E(X+c)=E(X)+c\]}
$E(X+c)=E(X)+E(c)\\
\quad \quad \quad \quad=E(X)+c$ (答)
\subsubsection{定数倍の期待値は,期待値の定数倍と等しい:\[E(cX)=cE(X)\]}
$E(cX)=E(c)E(X)\\
\quad \quad \quad=cE(X) $(答)

\subsection{宿題3}
\subsubsection{定数の分散はゼロ\[V(c)=0\]}
分散の定義より、Xが確率変数に場合には$V(X)=E(X-\mu_{X})^{2}$,\\
ここでcは定数なので、$E(c-c)^{2}=0$ (答)
\subsubsection{定数を足したものの分散は,もとの分散と等しい\[V(X+c)=V(X)\]}
V(X+c)=V(X)+V(c)= V(X) (答)
\subsubsection{定数倍の分散は,もとの分散に定数の2乗をかけたものと等しい\[V(cX)=c^{2}V(X)\]}
=$E(cX-c\mu_{Y})^{2}$\\
=$E(c^{2}(X-\mu_{Y})^{2})$\\
=$c^{2}E(X-\mu_{Y})^{2}$\\
=$c^{2}V(X)$(答)
\subsection{宿題4}
\subsubsection{二つの確率変数 と の和の期待値は,それぞれ
の期待値の和と等しい:\[\]}
$E(X+X^{'})=\\
\sum_{x}\sum_{x^{'}}(x+x^{'})P(X=x,X^{'}=x^{'})\\
=\sum_{x}\sum_{x^{'}}xP(X=x,X^{'}=x^{'})+\sum_{x}\sum_{x^{'}}x^{'}P(X=x,X^{'}=x^{'})\\
=\sum_{x}x\sum_{x^{'}}P(X=x,X^{'}=x^{'})+\sum_{x}x^{'}\sum_{x^{'}}P(X=x,X^{'}=x^{'})\\
=\sum_{x}xP(X=x)+\sum_{x^{'}}x^{'}P(X^{'}=x^{'})\\$
=$E(X)+E(X^{'})$(答)
\subsubsection{しかし と の和の分散は,一般にはそれぞれの分散の和とは等しくない\[\]}
$V(X+X^{'})=E((X+X^{'})-(\mu_{X}+\mu_{X^{'}}))^{2}\\
=E((X-\mu_{X})^{2}+(X^{'}-\mu_{X^{'}})^{2}))+2E(X-\mu_{X})(X^{'}-\mu_{X^{'}})\\
=V(X)+V(X^{'})+2Cov(X,X^{'})$(答)\\
\end{document}