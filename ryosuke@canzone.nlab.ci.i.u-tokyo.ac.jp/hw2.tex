\documentclass{article}
\begin{document}
\begin{center}
\section*{先端データ解析論 レポート第2回}
48196635 桑原亮介
\end{center}

\subsection*{宿題1}
目的:ガウスカーネルモデルに対する$l_{2}$-正則化を用いた最小二乗回帰の交差確認を実装し、正則化パラメタとガウス幅を決定する。\\
\noindent \\
方法:ガウス幅/正則化パラメタをそれぞれ$\{0.1,1.0,100.0\}$に設定した際の計9パターンで予測式を立て、正解データとの最小二乗誤差のもっとも低い組み合わせを決定する。\\
\noindent \\
結果:ガウス幅1.0、正則パラメタ0.1の際に誤差が最小となった。\\



\subsection*{宿題2}

\end{document}